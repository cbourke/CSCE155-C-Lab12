\documentclass[10pt]{exam}
%\printanswers
\usepackage{fullpage}

\setlength{\parindent}{0pt}
\setlength{\parskip}{.25cm}

\usepackage{graphicx}

\usepackage{xcolor}

\definecolor{darkred}{rgb}{0.5,0,0}
\definecolor{darkgreen}{rgb}{0,0.5,0}
\usepackage{hyperref}
\hypersetup{
  letterpaper,
  colorlinks,
  linkcolor=red,
  citecolor=darkgreen,
  menucolor=darkred,
  urlcolor=blue,
  pdfpagemode=none,
  pdftitle={Introduction To Git},
  pdfauthor={Christopher M. Bourke},
  pdfcreator={$ $Id: cv-us.tex,v 1.28 2009/01/01 00:00:00 cbourke Exp $ $},
  pdfsubject={PhD Thesis},
  pdfkeywords={}
}

\definecolor{MyDarkBlue}{rgb}{0,0.08,0.45}
\definecolor{MyDarkRed}{rgb}{0.45,0.08,0}
\definecolor{MyDarkGreen}{rgb}{0.08,0.45,0.08}

\definecolor{mintedBackground}{rgb}{0.95,0.95,0.95}
\definecolor{mintedInlineBackground}{rgb}{.90,.90,1}

%\usepackage{newfloat}
\usepackage[newfloat=true]{minted}
\setminted{mathescape,
               linenos,
               autogobble,
               frame=none,
               framesep=2mm,
               framerule=0.4pt,
               %label=foo,
               xleftmargin=2em,
               xrightmargin=0em,
               startinline=true,  %PHP only, allow it to omit the PHP Tags *** with this option, variables using dollar sign in comments are treated as latex math
               numbersep=10pt, %gap between line numbers and start of line
               style=default, %syntax highlighting style, default is "default"
               			    %gallery: http://help.farbox.com/pygments.html
			    	    %list available: pygmentize -L styles
               bgcolor=mintedBackground} %prevents breaking across pages
               
\setmintedinline{bgcolor={mintedBackground}}
\setminted[text]{bgcolor={mintedBackground},linenos=false,autogobble,xleftmargin=1em}
%\setminted[php]{bgcolor=mintedBackgroundPHP} %startinline=True}
\SetupFloatingEnvironment{listing}{name=Code Sample}
\SetupFloatingEnvironment{listing}{listname=List of Code Samples}

\begin{document}

\section*{CSCE 155 - Lab 12 - Recursion - Worksheet}

Names: \underline{\hspace{10cm}}

\begin{questions}

\question Test your palindrome program on the following inputs:
\begin{parts}
  \part \mintinline{text}{abba}
  \part \mintinline{text}{aaba}
  \part \mintinline{text}{foobar}
  \part \mintinline{text}{civic}
  \part \mintinline{text}{civil}
  \part \mintinline{text}{foo}
\end{parts}

\question Fill out the following table with the specified values using the \mintinline{text}{recursiveFunction} program.

\begin{table}[h]
\centering
{\renewcommand{\arraystretch}{1.35}
\begin{tabular}{|c|p{4cm}|p{4cm}|}
\hline
$n$ & $f(n)$ & Time (sec) \\
\hline\hline
0   &        &            \\
\hline
1   &        &            \\
\hline
2   &        &            \\
\hline
5   &        &            \\
\hline
10  &        &            \\
\hline
20  &        &            \\
\hline
30  &        &            \\
\hline
40  &        &            \\
\hline
42  &        &            \\
\hline
44  &        &            \\
\hline
46  &        &            \\
\hline
48  &        &           \\
\hline
\end{tabular}
}
\end{table}

\question Without actually running it, make a prediction on how long 
  it would take to run the program for $n = 50$.  How long would it 
  run for $n = 100$?

\question What values do you get for each of the following inputs with 
  your Jacobsthal program?
\begin{parts}
  \part $n = 0$
  \part $n = 1$
  \part $n = 5$
  \part $n = 10$
  \part $n = 20$
  \part $n = 32$
\end{parts}

\question Demonstrate your working programs to a lab instructor.  

\end{questions}
  
Lab Instructor Signature\underline{\hspace{7.5cm}}

\end{document}
