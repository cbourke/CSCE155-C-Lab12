\documentclass[12pt]{scrartcl}


\setlength{\parindent}{0pt}
\setlength{\parskip}{.25cm}

\usepackage{graphicx}

\usepackage{xcolor}

\definecolor{darkred}{rgb}{0.5,0,0}
\definecolor{darkgreen}{rgb}{0,0.5,0}
\usepackage{hyperref}
\hypersetup{
  letterpaper,
  colorlinks,
  linkcolor=red,
  citecolor=darkgreen,
  menucolor=darkred,
  urlcolor=blue,
  pdfpagemode=none,
  pdftitle={Introduction To Git},
  pdfauthor={Christopher M. Bourke},
  pdfcreator={$ $Id: cv-us.tex,v 1.28 2009/01/01 00:00:00 cbourke Exp $ $},
  pdfsubject={PhD Thesis},
  pdfkeywords={}
}

\definecolor{MyDarkBlue}{rgb}{0,0.08,0.45}
\definecolor{MyDarkRed}{rgb}{0.45,0.08,0}
\definecolor{MyDarkGreen}{rgb}{0.08,0.45,0.08}

\definecolor{mintedBackground}{rgb}{0.95,0.95,0.95}
\definecolor{mintedInlineBackground}{rgb}{.90,.90,1}

%\usepackage{newfloat}
\usepackage[newfloat=true]{minted}
\setminted{mathescape,
               linenos,
               autogobble,
               frame=none,
               framesep=2mm,
               framerule=0.4pt,
               %label=foo,
               xleftmargin=2em,
               xrightmargin=0em,
               startinline=true,  %PHP only, allow it to omit the PHP Tags *** with this option, variables using dollar sign in comments are treated as latex math
               numbersep=10pt, %gap between line numbers and start of line
               style=default, %syntax highlighting style, default is "default"
               			    %gallery: http://help.farbox.com/pygments.html
			    	    %list available: pygmentize -L styles
               bgcolor=mintedBackground} %prevents breaking across pages
               
\setmintedinline{bgcolor={mintedBackground}}
\setminted[text]{bgcolor={mintedBackground},linenos=false,autogobble,xleftmargin=1em}
%\setminted[php]{bgcolor=mintedBackgroundPHP} %startinline=True}
\SetupFloatingEnvironment{listing}{name=Code Sample}
\SetupFloatingEnvironment{listing}{listname=List of Code Samples}


\title{CSCE 155 - C}
\subtitle{Lab 12 - Recursion}
\author{Dr.\ Chris Bourke}
\date{~}

\begin{document}

\maketitle

\section*{Prior to Lab}

Before attending this lab:
\begin{enumerate}
  \item Read and familiarize yourself with this handout.
  \item Review the following free textbook resources:
	\begin{itemize}
  	  \item \url{http://en.wikibooks.org/wiki/C_Programming/Procedures_and_functions#Recursion}
	\end{itemize}
\end{enumerate}

\section*{Peer Programming Pair-Up}

To encourage collaboration and a team environment, labs will be
structured in a \emph{pair programming} setup.  At the start of
each lab, you will be randomly paired up with another student 
(conflicts such as absences will be dealt with by the lab instructor).
One of you will be designated the \emph{driver} and the other
the \emph{navigator}.  

The navigator will be responsible for reading the instructions and
telling the driver what to do next.  The driver will be in charge of the
keyboard and workstation.  Both driver and navigator are responsible
for suggesting fixes and solutions together.  Neither the navigator
nor the driver is ``in charge.''  Beyond your immediate pairing, you
are encouraged to help and interact and with other pairs in the lab.

Each week you should alternate: if you were a driver last week, 
be a navigator next, etc.  Resolve any issues (you were both drivers
last week) within your pair.  Ask the lab instructor to resolve issues
only when you cannot come to a consensus.  

Because of the peer programming setup of labs, it is absolutely 
essential that you complete any pre-lab activities and familiarize
yourself with the handouts prior to coming to lab.  Failure to do
so will negatively impact your ability to collaborate and work with 
others which may mean that you will not be able to complete the
lab.  

\section{Lab Objectives \& Topics}
At the end of this lab you should be familiar with the following
\begin{itemize}
  \item How to use recursion to solve a problem
  \item Understanding the pitfalls of recursion
\end{itemize}

\section{Background}

In a programming language, recursion is a mechanism by which 
a function or method repeatedly calls itself.  Recursion can be a 
powerful tool in a programming language, lending itself to a 
divide-and-conquer strategy to problem solving using very simple 
code.  However, there are draw backs as we have previously 
demonstrated.  Moreover, a non-recursive solution is always 
possible and may be far more efficient when the proper data 
structures are used.  In general, a recursive function should:
\begin{enumerate}
  \item Define a \emph{base case} which does not call the 
	function again, but provides a direct solution.
  \item Otherwise, recursively call itself on a smaller input that 
	works toward the base case. 
\end{enumerate}
	
In this lab, you will use recursion to solve a couple of problems.

\section{Activities}

Clone the project code for this lab from GitHub using the following 
URL: \url{https://github.com/cbourke/CSCE155-C-Lab12}

\subsection{Solving a Problem Using Recursion}

A \emph{palindrome} is a string that is the same backward and 
forward: ``rats live on no evil star'' or single words such as ``civic'' 
or ``deed'' are palindromes.  In this activity you will design and 
implement a recursive function that determines whether or not 
a given string is a palindrome.  The recursive function should 
work as follows.  
\begin{itemize}
  \item The function is given a string, a left-index, and a right-index.
  \item If the characters at the left index and right index are not 
	equal then we can stop, it is not a palindrome.
  \item Otherwise, we need to check the substring not including the 
	(equal) characters at the left/right indices.
  \item Until the string is ``empty'' (the left/right indices are the same 
	or have crossed each other) at which point we can say, yes it 
	is a palindrome
\end{itemize}

For example, given the word ``civic'' we would check the first and 
last characters.  Finding them the same, we would recursively call 
the function on the substring ``ivi''.  However, with the proper function 
parameters, we don't create a new substring; we simply pass the 
relevant indices that define the substring.

\subsubsection*{Instructions}

\begin{enumerate}
  \item Complete the \mintinline{c}{isPalindrome} function in the 
	\mintinline{text}{palindrome.c} source file as specified above.
  \item Compile and test your program from the command line.  
	Note: you can test a phrase containing whitespace by encapsulating 
	it in double quotes; example: \\
	\mintinline{text}{./a.out "rats live on no evil star"}
  \item Answer the questions on your worksheet and demonstrate your 
  	working program to a lab instructor
\end{enumerate}
	
\subsection{A Recursive Function}

We have provided you a program, \mintinline{text}{recursiveFunction.c} 
that computes a recursive function similar to the Fibonacci sequence.  
This function, however, is defined as follows: 

$$f(n) = \left\{
\begin{array}{ll}
2 & \textrm{if } n = 0 \\
2 & \textrm{if } n = 1 \\
f(n-1) + \lfloor f(n-2) / 2 \rfloor & \textrm{otherwise}
\end{array}
\right.$$

Examine at the contents of this source file and understand what is 
going on.  Compile and run this program which accepts $n$ as a 
command line argument.  Answer the questions on your worksheet.

\subsection{The Jacobsthal Function}

The Jacobsthal sequence is very similar to the Fibonacci sequence 
in that it is defined by its two previous terms.  The difference is that 
the second term is multiplied by two.  
$$J_n = \left\{
\begin{array}{ll}
0 & \textrm{if } n = 0 \\
1 & \textrm{if } n = 1 \\
J_{n-1} + 2J_{n-2} & \textrm{otherwise}
\end{array}
\right.$$

Write a recursive function to compute the $n$-th Jacobsthal number.  
Write a \mintinline{c}{main} function that reads in an integer $n$ and 
outputs $J_n$.  Test your program for $n = 32$, the largest Jacobsthal 
number expressible by a 32-bit signed 2-s complement integer.  
Answer the questions on your worksheet.

\subsection{String Suffixes}

A suffix tree is a data structure that holds representations of all the 
suffixes of a particular string.  For example, ``computer'' has eight 
suffixes, ``computer'', ``omputer'', $\ldots$, ``ter'', ``er'', ``r''.  Suffix 
tree data structures are used extensively in DNA analysis and other 
pattern matching algorithms.  For this exercise, we will not be 
generating an actual suffix tree.  Rather, we will simply write a 
recursive function that, given a string outputs all the suffixes.  
Your program should take a string from the command line and 
produce a list of suffixes, one per line.  An example run: 

\begin{minted}{text}
>a.out computer
computer
omputer
mputer
puter
uter
ter
er
r
\end{minted}

Demonstrate your working program to a lab instructor and have them 
sign off on your worksheet.

\section{Advanced Activity (Optional)}

\begin{enumerate}
  \item Memoization is a technique that is used to avoid repeated 
  	recursive calls on the same value.  The basic idea is that intermediate 
	values are stored in a tableau.  Upon a recursive call, the tableau is 
	checked: if it contains a valid value, that value is used, otherwise the 
	recursion is performed.  When a value has been computed, it is added 
	to the table so that subsequent recursive calls can use it and avoid 
	repeating work.  Redesign your Jacobsthal function to use memoization 
	in its recursion.
  \item Modify your recursive suffix function as follows.  Instead of printing 
	each suffix, store it in an array of strings.  Think about who should be 
	responsible for setting up the array--the caller or the actual function? 
\end{enumerate}
	
\end{document}
